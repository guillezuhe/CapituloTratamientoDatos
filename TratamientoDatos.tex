\documentclass{book}
\usepackage{hyperref}
\usepackage{amsmath}
% \usepackage{xcolor}
\usepackage[dvipsnames]{xcolor}
\begin{document}

\chapter*{Tratamiento de datos}

En este capítulo se recoge un breve resumen de tratamiento y representación de datos,
con los elementos fundamentales necesarios para la elaboración de informes del curso.

\section{Expresión de resultados}

Una medición nos permite determinar un valor asociado a una magnitud física, bien mediante
la cuenta de un número de sucesos o comparación con una unidad de medida. Cuando realizamos
una medición es imposible determinar el valor exacto de dicha magnitud, sino que debemos
limitarnos a encontrar un valor aproximado siempre limitado por la precisión de los instrumentos
de medida y el observador, así como las propiedades intrínsecas de la materia (fluctuaciones,
indeterminación, etc). De esta manera, cuando realizamos sucesivas medidas podemos obtener valores
ligeramente distintos, por lo que presentaremos nuestros resultados dando un valor estimado dentro
de un rango en el que tenemos un grado de confianza en el que se encuentra el valor verdadero de
la magnitud. Esto es, toda magnitud física derivada de un experimento lleva asociado un error, 
que ha de expresarse siempre junto al valor estimado, normalmente mediante el símbolo '$\pm$'.
Sea una magnitud $x$, denotaremos su error asociado como $\Delta x$. Veamos un ejemplo:


\begin{equation}
    m = 5.64 \pm 0.23 \; \mathrm{kg}
\end{equation}

Destacamos varios puntos sobre la representación de este error asociado:

\begin{itemize}
  \item Todos los resultados presentes en un informe de laboratorio deben incluir su error 
  asociando.
  \item Todos los resultados deben expresarse junto a las \textbf{unidades} en las que se mide dicha 
  magnitud física.
  \item Todo error tiene como \textbf{máximo dos cifras significativas}. Además, la cantidad medida
  ha de tener como última cifra significativa la última de su error. No tiene sentido proporcionar
  un valor con más precisión que el error (Ver ejemplos a continuación.)
  \item Frecuentemente, las variables se representan con letra formateada cursiva, mientras 
  que para las unidades se emplea un formato
\end{itemize}

A continuación se muestran algunos ejemplos de buena y mala representación de resultados medidos.

\begin{equation}
  \begin{aligned}
    & \color{ForestGreen} P = 14800 \pm 3000 \; \mathrm{Pa} \\
    & \color{ForestGreen} \rho = 7860.1 \pm 2.3 \; \mathrm{kg / m^{-3}} \\
    & \color{ForestGreen} v = 2300 \pm 700 \; \mathrm{m/s} \\
    & \color {ForestGreen} d = ( 23.21 \pm 75 ) \pm 10^{-6} \; \mathrm{\mu m}
  \end{aligned}
\end{equation}

\begin{equation}
  \begin{aligned}
    & \color{Red} P = 14845 \pm 3253 \; \mathrm{Pa} \\
    & \color{Red} \rho = 7860.1 \pm 2.3 \; kg / m^{-3} \\
    & \color{Red} v = 2336 \pm 700 \; \mathrm{m/s} \\
    & \color {Red} d = ( 23.2186 \pm 7506 ) \pm 10^{-6} \; \mathrm{\mu m}
  \end{aligned}
\end{equation}


Es muy importante representar los resultados correctamente. Un informe nunca podrá obtener una
buena calificación si hay fallos en este aspecto. Como norma general, se recomienda trabajar con
unidades del sistema internacional y sus múltiplos para evitar errores derivados del manejo de
unidades.

En las siguientes secciones se detalla cómo calcular el error asociado a una magnitud física. 
Este error se obtendrá de forma diferente dependiendo de si el valor se observa directamente en
instrumento de medida (\textbf{medida directa}) o si se obtiene utilizando expresiones matemáticas a partir
de una o varias medidas directas (\textbf{medida indirecta}).


\section{Incertidumbre en medidas directas}
Esta incertidumbre se asocia con la medida directa de una magnitud utilizando un instrumento.
Dentro de esta incertidumbre, encontramos contribuciones de distinto tipo que debemos considerar
en su conjunto.
\subsection{Tipo A}
También conocido como incertidumbre estadística. Se asocia con factores ambientales aleatorios, 
fluctuaciones o vibraciones. Por ejemplo, variaciones de temperatura a lo largo de un experimento.
Para determinar esta contribución al error debemos realizar $n$ medidas, y se calcula su desviación
típica $s$. La magnitud tomará el valor medio de estas $n$ medidas, y el error asociado de tipo A 
vendrá determinado por:

\begin{equation}
  \Delta x_A = \frac{s}{\sqrt{n}}
\end{equation}

\textcolor{red}{¿Incluimos multiplicar por la t de student o preferís no hacerlo?}

\subsection{Tipo B}
También conocida como incertidumbre sistemática de precisión, se relaciona con la sensibilidad,
denotada como $\delta x$, del instrumento utilizado. Así, el error tipo B se corresponderá con:

\begin{equation}
  \Delta x_B = \delta x
\end{equation}

En caso de no disponer de ninguna referencia, suele tomarse su valor como una unidad de la cifra 
que se puede apreciar con el aparato. Sin embargo, el error instrumental suele ser mayor que esta 
última cifra y muchos fabricantes así lo indican en los manuales de uso de los dispositivos. Por 
tanto es necesario consultar siempre si el instrumento que estamos utilizando dispone de alguna 
indicación de su precisión.

Un ejemplo muy claro es el caso de los polímetros, en los que siempre se dispone de un manual en
el que el fabricante detalla el error instrumental asociado a cada magnitud como un tanto por
ciento de la medida más algunos dígitos. Por ejemplo, supongamos que medimos un volataje y
la pantalla del polímetro muestra $23.85 \; \textrm{V}$. Si recurrimos al manual encontraremos que la precisión
del aparato para el voltaje es de un 1\% del valor leído + 3 dígitos. Por tanto, el error de tipo
B asociado a esta medida será

\begin{equation}
  \Delta x_B = 0.23 \pm 0.03 \; \textrm{V} = 0.26 \; \textrm{V}
\end{equation}

\subsection{Incertidumbre combinada y otras contribuciones}

Las dos contribuciones anteriores han de considerarse para el cálculo del error total asociado a
una media. Es por ello que se define la incertidumbre combinada como la suma cuadrática de las
incertidumbres de tipo A y B.

\begin{equation}
  \Delta x_C = \sqrt{(\Delta x_A)^2 + (\Delta x_B)^2}
\end{equation}

No obstante, existen otras causas distintas a las anteriores que pueden degradar una medida, y
que agrupamos bajo el nombre de \textbf{incertidumbre sistemática}. Estas son incertidumbres que se producen
en cada medida de la magnitud que se realiza, y que pueden deberse a una mala interpretación de
los valores proporcionados por el dispositivo o a un error de calibración (error de cero) que
desplace o distorsione la medida. Hemos de asegurarnos antes de comenzar la toma de datos que no
estamos incurriendo en ningún error de este tipo.

\section{Incertidumbre en medidas indirectas}

Una vez obtenidas las incertidumbres de las medidas directas, si queremos determinar cualquier
magnitud derivada, debemos obtener el error de esta medida indirecta. Supongamos una medida
indirecta $y$ que se obtiene a partir de varias medidas directas independientes $x_1$, $x_2$, 
... $x_n$  y una serie de constntes $c_1$, $c_2$, ... $c_m$, mediante una relación funcional:

\begin{equation}
  y = f(x_1, x_2, ... c_1, c_2, ...)
\end{equation}

El valor esperado de esta magnitud indirecta $y$ se calcula a partir de los valores medidos:

\begin{equation}
  \bar{y} = f(\bar{x}_1, \bar{x}_2, ... c_1, c_2, ...)
\end{equation}

La incertidumbre de esta magnitud $y$ se obtiene de la llamada \textbf{propagación cuadrática de 
errores}.

\begin{equation}
  \Delta y = \sqrt{ \left( \frac{\partial f}{\partial x_1} \right)^2 (\Delta_C x_1)^2 + 
  \left( \frac{\partial f}{\partial x_2} \right)^2 (\Delta_C x_2)^2 + ... +
  \left( \frac{\partial f}{\partial x_n} \right)^2 (\Delta_C x_n)^2}
\end{equation}

Respecto a las constantes, debemos de asegurarnos de tomar sus valores con la suficiente precisión
como para que su contribución a la incertidumbre pueda considerarse despreciable.

\section{Ajustes y representaciones}

\subsection{Test de bondad del ajuste}









\end{document}